\pagenumbering{arabic}
\chapter*{Introduction}
\label{ch:intro}
\addcontentsline{toc}{chapter}{Introduction}

The study of elementary particles and their behaviour relies on a great many
sources of experimental information.  In order to verify the predictions of the
Standard Model (SM) of particle physics or indeed extensions beyond, precise
and accurate measurements must be made of the fundamental properties of matter.
Building upon decades of advances in the study of elementary particles, today
the foremost source of cutting edge measurements in particle physics is the
Large Hadron Collider (LHC) based at CERN in Switzerland.  The LHC, through
colossal scientific and human effort has opened up the study of the properties
of nature to scales that were previously inaccessible. 

The LHC probes the building blocks of nature by the collision of high energy
protons. Maximising the physics potential of the LHC therefore requires a deep
understanding of the composite nature of the proton. The short range dynamics
of a proton's constituent particles can be described by perturbative Quantum
Chromo-Dynamics (QCD), however an understanding of the low energy behaviour is
impossible to obtain through perturbative methods, therefore making its
determination by a calculation from first principles challenging. In practice
the structure of the proton is understood through a comprehensive analysis of
experimental data, and described in terms of Parton Distribution Functions
(PDFs). These functions parametrise the unknown non-perturbative dynamics of
the proton. As a universal property of protons, the PDFs may be determined from
available experimental data and then applied in the calculation of predictions
for other experiments, therefore making the application of QCD in hadron
collisions into a predictive theory which may be tested via comparison to
data.

The accurate determination of parton densities in the proton is an ongoing
effort, with several groups providing competing sets of parton distributions.
The NNPDF collaboration provides a set of parton distributions determined
through a rather different methodology than the standard procedures, resulting
in a PDF set suffering from little of the parametrisation bias possible in
competing approaches. Furthermore the NNPDF methodology has a unique treatment
of the experimental uncertainty propagation, leading to a statistically sound
estimation of the uncertainties in the resulting PDFs.

While a precise knowledge of the dynamics of the proton is vital for LHC studies of physics in the standard model and beyond, LHC data also has the potential to provide the most in depth information on parton densities to date. This thesis is based upon work conducted in the study of early LHC standard model measurements of particular sensitivity to parton distributions. The inclusion of such an experimental dataset into a fit in the NNPDF framework has necessitated the development of a number of tools for the efficient calculation of collider observables. These tools and their applications shall be discussed alongside the development of the NNPDF methodology to better handle the ever-enlarging LHC dataset.

This thesis is arranged as so. In Chapter One we shall provide a brief discussion of the theoretical structure of parton distributions, where they arise in the calculation of deep-inelastic scattering cross-sections and further theoretical background relevant to the reliable determination of PDFs from experimental data. Chapter Two is concerned with the practical extraction of PDFs and shall describe experimental observables of interest along with the different approaches used by major PDF collaborations to fit the data. In Chapter Three, the tools that have been developed to enable the inclusion of a large LHC dataset into a computationally intensive fit such as the NNPDF procedure are introduced and described. A brief summary of experimental measurements at the LHC of interest to the determination of PDFs is provided in Chapter Four. In Chapter Five, we shall examine the impact of some of these measurements made by LHC collaborations upon PDF determinations, enabled by the tools developed in Chapter Three. The data impact will be assessed in the context of the two most recent public releases of the NNPDF collaboration; providing a summary of their datasets and the tools used in their extraction. Finally in Chapter Six we examine some of the methodological improvements that have been made in the NNPDF procedure in order to ensure the maximal efficiency in extracting new information on PDFs from future LHC measurements, and demonstrate their application in early prototypes of the NNPDF3.0 PDF set.


